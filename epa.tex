%% LyX 2.2.2 created this file.  For more info, see http://www.lyx.org/.
%% Do not edit unless you really know what you are doing.
\documentclass[english]{scrartcl}
\usepackage[T1]{fontenc}
\usepackage[latin9]{inputenc}
\usepackage{color}
\usepackage{babel}
\usepackage[unicode=true,pdfusetitle,
 bookmarks=true,bookmarksnumbered=false,bookmarksopen=false,
 breaklinks=true,pdfborder={0 0 0},pdfborderstyle={},backref=false,colorlinks=true]
 {hyperref}
\hypersetup{
 linkcolor=blue,citecolor=blue,urlcolor=blue}

\makeatletter
%%%%%%%%%%%%%%%%%%%%%%%%%%%%%% User specified LaTeX commands.
\usepackage{mciteplus}

\makeatother

\begin{document}

\title{Arguments for a Section 18 Emergency Exemption to Allow Use of \emph{Metarhizium
majus} and \emph{Oryctes} nudivirus as Biological Control Agents for
Coconut Rhinoceros Beetle on Guam}

\author{Aubrey Moore}
\maketitle
\begin{enumerate}
\item Coconut rhinoceros beetle (CRB), \emph{Oryctes rhinoceros}, a major
pest of coconut and other palms, was first detected on Guam in 2007.
Adults bore into the crowns of palms to feed on sap, causing mortality
if they destroy the meristem. CRB grubs do no economic damage. They
feed in dead coconut stems, fallen logs, and any material high in
organic content such as green waste, manure, sawdust, and soil. Guam
is currently experiencing an uncontrolled and unmonitored CRB outbreak
triggered by abundant breeding sites left in the wake of a recent
typhoon. Many palms have already been killed island-wide, \href{https://youtu.be/_doMxC5ACes}{approaching 100\%{} mortality in some areas}.
If the current outbreak cannot be controlled, it is likely that Guam
will loose 50\% or more of its palms, as happened when Palau was invaded
by CRB at the end of WWII. It is also likely that CRB will be accidentally
exported to other islands. The CRB outbreak prompted Governor Calvo
to declare a \href{http://kuam.images.worldnow.com/library/08d5c829-0405-411f-9e49-6dc4a1b83669.pdf}{state of emergency}
on July 13, 2017.
\item The insect pathogens, \emph{Oryctes} nudivirus, OrNV, and \emph{Metarhizium
majus}, also known as green muscardine fungus, GMF, are biological
control agents for the \emph{Oryctes rhinoceros}, the coconut rhinoceros
beetle. OrNV is applied by autodissemination as a classical, innoculative
biocontrol agent by autodissemination. GMF is used as an augmentive
biocontrol agent by incorporating spores into active or potential
CRB breeding sites.
\item Humans have lived with \emph{Metarhizium majus}, \emph{Oryctes nudivirus}
hundreds of thousands of years. No adverse human health effects have
been reported. These insect pathogens are not genetically engineered
organisms nor are they synthetic pesticides. They have always been
part of the human environment.
\item The total amount of active ingredient applied to the environment is
minuscule. In an island-wide biocontrol program using GMF, a few milligrams
of spores will be used. In the case of OrNV, a few nanograms of virus
particles will be used. 
\item Both \emph{M. majus} and OrNV are known to attack only dynastid beetles.
The coconut rhinoceros beetle is the only member of Dynastidae on
Guam. This high degree of host specificity is highly desired for insect
pathogens used as biocontrol agents.
\item There are no reports of nontarget effects following release of \emph{M.
majus} and OrNV as biocontrol agents on Guam, on other Pacific islands
and elsewhere.
\item Perceived, potential risks to human health and the environment from
exposure to \emph{M. majus} and OrNV should be balanced against the
known risks to human health and the environment from the loss of coconut
palms.
\item \emph{M. majus} and OrNV are regulated as biocontrol agents by USDA-APHIS.
Perhaps regulation of these same biocontrol agents by USEPA is an
unnecessary duplication of effort. Both \emph{M. majus} and OrNV were
imported to Guam and released under conditions of permits received
from USDA-APHIS. 
\end{enumerate}

\end{document}
